\documentclass{article}
\usepackage[utf8]{inputenc}
\usepackage[russian]{babel}
\usepackage{graphicx}

\graphicspath{ {./images/} }

%%% Size definition (do not edit!)
\textwidth=115mm % ???
\textheight=40\baselineskip % 170mm
\footskip=5mm

\setcounter{page}{1}

\newtheorem{lemma}{Лемма}[section]
\newtheorem{remark}{Примечание}[section]
\newtheorem{theorem}{Теоpема}[section]
\newtheorem{statement}{Утвеpждение}[section]
\newtheorem{definition}{Опpеделение}[section]

\def\proof{\par\noindent{\bf Доказательство}. \ignorespaces}
\def\endproof{{\ \vbox{\hrule\hbox{%
\vrule height1.3ex\hskip1.3ex\vrule}\hrule
}}\vspace{2mm}\par}

\newcommand{\bft}{\mbox{}\vspace*{-15mm}\mbox{}\\{}\bf{}}
\date{}

%%%%%%%%%%%%%%%%%%%%%%%%%%%%%%%%%%%%%%%%%%%%%%%%%%%%%%%%%%%%%%%%%%%%%%%%%%%%%

\title{\bft Библиотека для морфологического анализа свободнообразуемых слов русского языка.}
\author{E.\,Е.\,Гончаренко\footnotemark[1]\\[2mm]
\footnotemark[1]~{Факультет ВМК МГУ, gee.github@gmail.com}\\~~~}

\begin{document}
\maketitle

%%%%%%%%%%%%%%%%%%%%%%%%%%%%%%%%%%%%%%%%%%%%%%%%%%%%%%%%%%%%%%%%%%%%%%%%%%%%%
\section*{Введение}

\

Задачей лингвистического процессора является преобразование \\
естественно-языкового предложения (или даже целого текста) в некоторый набор семантических структур, являющихся формальным представлением “смысла” исходного предложения или текста. Цель такого преобразования — обеспечить исходные данные для работы поисковых механизмов СУБД.
Вот список тех задач, в которых можно использовать лингвистический процессор: 
\begin{enumerate} 
    \item написание переводчика;
    \item задачи распознавания и синтеза речи;
    \item распознавание текста;
    \item проверка орфографии;
    \item проверка синтаксиса;
    \item информационно-поисковые системы;
\end{enumerate}

\newpage

Основным недостатком существующих лингвистических процессоров является чрезмерно большой объем словаря, порождающий ряд технических проблем:
\begin{itemize}
    \item большие затраты труда на создание и поддержание словаря;
    \item невозможность полного размещения словаря в оперативной памяти компьютера при анализе;
    \item высокая избыточность информации, связанной с постоянными  \\ признаками каждой словоформы (морфологическими, синтаксическими, семантическими);
\end{itemize}

Современные компьютерные программы, анализирующие текст на естественном языке, как правило, используют словари. Цель словарей помочь распознать встреченную текстовую цепочку.

Целью данной работы является создание программы, которая, используя реально существующую лингвистическую базу данных, выдает морфологические характеристики некоторых слов, не содержащихся в этой базе данных.
Программа основана на использовании морфологического анализа структуры незнакомого слова.
Приблизительно анализ слова работает в такой последовательности. От предлагаемого слова отрезаются возможные префиксы, и оставшаяся часть проверяется на наличие в лингвистической базе данных. Если оставшаяся часть слова присутствует в базе данных, то в качестве информации об исходном слове, выдается полученная информация о части слова, с учетом всех префиксов.


%%%%%%%%%%%%%%%%%%%%%%%%%%%%%%%%%%%%%%%%%%%%%%%%%%%%%%%%%%%%%%%%%%%%%%%%%%%%%
\section{Постановка задачи}

Целью данной работы является создание программы, которая, используя реально существующую лингвистическую базу данных, выдает морфологические характеристики для следующих классов слов:
\begin{itemize}
    \item свободнообразуемые слова;
    \item слова с дефисом;
    \item сложные слова.
\end{itemize}

\newpage

Свободнообразуемые слова должны удовлетворять следующим условиям:
\begin{itemize}
    \item Стандартность их соединения с существительными и прилагательными.
    \item Стандартность значения.
    \item Структурная самостоятельность.
\end{itemize}

Программа работает следующим образом: \\
От предлагаемого слова отрезаются возможные префиксы, и оставшаяся часть проверяется на наличие в лингвистической базе данных. Если оставшаяся часть слова присутствует в базе данных, то в качестве информации об исходном слове, выдается полученная информация (падеж, склонение и т.д.) о части слова, с учетом всех префиксов. Программа автоматически меняет основу, ударную букву, ставит второстепенное ударение. 

\section{Алгоритм анализа слова.}

Анализ слова сводится к следующей последовательности действий:

\begin{enumerate} 
    \item На вход процедуры подается слово Х.
    \item Пытаемся найти в слове Х дефис. Если мы его нашли, то ту часть слова Х, которая была до дефиса (включая дефис), мы сохраняем как Х1; а оставшуюся часть слова Х как Х2 и переходим к шагу 5, иначе на шаг 3.
    \item Если слово Х начинается на префикс из списка префиксов (см. пункт: Словарная информация для базы данных), то мы сохраняем этот префикс как Х1, а оставшуюся часть слова Х как Х2 и переходим к шагу 5, иначе на шаг 4.
    \item Если слово Х начинается на порядковое числительное (например: тысячетрехсотдвадцатичетырехдневный), то мы сохраняем это \\числительное как Х1, а оставшуюся часть слова Х как Х2 и переходим к шагу 5, иначе на шаг 8.
    \item Обращаемся к морфологическому анализатору со словом Х2. Если морфологический анализатор выдал морфологические характеристики слова Х2, то перейти на шаг 6. Если же морфологический анализатор выдал, что слово не найдено, то перейти на шаг 7.
    \item В качестве информации о слове Х, выдается информация о слове Х2, модифицированная следующим образом:
    \begin{itemize}
        \item к основе слова Х2 слева приписывается слово Х1;
    \end{itemize}
    \begin{itemize}
        \item меняется номер ударной буквы;
    \end{itemize}
    \begin{itemize}
        \item ставится второстепенное ударение.
    \end{itemize}
    \item Рекурсивно вызываем данный алгоритм для слова Х2. Если алгоритм выдал морфологические характеристики слова Х2, то перейти на шаг 6. Если же алгоритм выдал, что слово не найдено, то перейти на шаг 8.
    \item Слово не распознано. Стоп.
\end{enumerate}

\end{document}
